\section{Background}

\subsection {Aqueous Surface Location}

	The analyses performed in this study relate results to the location of the gas-liquid interface. An interface, however, is a dynamic surface that is neither flat nor stationary. Previous studies have used the technique of fitting a lineshape to the density profile of the water or solute to extract interfacial shape and location parameters.\cite{Shamay2010,Wick2006c,Chowdhary2006} Hyperbolic tangent functions have been used for fitting, and values for the ``Gibb's dividing surface'' location, and interfacial width are determined.\cite{Matsumoto1988} However, in long simulations the location of the interface will change, and the net motion of surface waters will alter the interfacial width at any given timestep. Thus, the averaged density profile fitting will capture average widths and locations, not instantaneous values. The interface can be characterized by density profiles, molecular orientational profiles, or interfacial intermolecular correlations. In this work we approach the problem by finding a reference location for each timestep by averaging the positions of the waters contained in the topmost monolayer. This provides us with a consistent reference point in the simulations to which we relate our analyses. This method is independent of the various measures of the interface, but still provides an intuitive point of reference.

\subsection{Analysis of Orientation}

	The simulated systems were analyzed to characterize the orientation of \wat~and \suldiox~in various environments above, within, and below the surface region. The orientation of the surface molecules as \suldiox~adsorbs and interacts with water is a property of any hydrate complex formation that takes places. Better understanding of the molecular orientational distributions will help elucidate the chemistry occurring during the adsorption process.

	The two molecules studied, \wat~and \suldiox, and similarly shaped with a C2$_v$ axis along their bisectors, and a molecular plane is defined by their three atoms. An body-fixed frame is defined for \wat~and \suldiox~as shown in figure \ref{fig:molecular-frame}. In each analysis a world-fixed reference axis is used that is the long axis of the system's periodic cell (normal to the plane of the water surface) unless otherwise noted. The orientational analyses presented herein focus on two angles used to define molecular orientation in space. 
	
	A ``tilt'' angle, $\theta$, defines the angle formed between the bisector vector (the molecular z-axis, pointing from the central atom in the direction of the other two atoms) and the positive system reference axis. Thus the value of $\theta$ falls within a range of $[0,\pi]$. An angle of $\theta=0$ indicates a molecule with its bisector aligned with the reference axis, while $\theta=\pi$ results from an anti-aligned configuration. 
	
	A second angle, $\phi$, defines the molecular ``twist''. $\phi$ is the angle formed between the vector normal to the plane of the molecule (molecular y-axis) and the system reference axis. The values of $\phi$ fall in the interval $[-\pi,\pi]$. Given certain values of $\theta$, $\phi$ provides additional information about whether the molecular orientation is ``flat'' to the surface (e.g. the plane of the molecule is aligned with the plane of the surface), or if it is perpendicular. %However, the symmetry of the \wat~and \suldiox~molecules create equivalence between the two H, or O, atoms (in the \wat~or\suldiox, respectively). $\phi=-\pi$ is equivalent to $\phi=\pi$, and both situations are equivalent to $\phi=0$. Because $\phi=-\phi$, the value is reported within the range of $[0,\frac \pi 2]$. 
	Figure \ref{fig:water-angles} shows the angle definitions relating the body-fixed axes to a system-fixed reference axis. 

%	The two molecules studied, \wat~and \suldiox, and similarly shaped with a C2$_v$ axis along their bisectors, and a molecular plane is defined by their three atoms. The orientational analyses presented herein focus on two angles used to define the molecular orientation in space. In each analysis a reference axis is used to define molecular angles, and is the z-axis unless otherwise specified. A ``tilt'' angle, $\theta$, defines the angle formed between a vector (generally the bisector vector pointing from the central atom in the direction of the other two atoms) and the positive reference axis. Thus the value of $\theta$ falls within a range of $[0,\pi]$. A second angle, $\phi$, defines the molecular ``twist'' around the reference axis. $\phi$ is formed between the projection of a vector onto the x-y reference plane, and the x-axis. $\phi$ is thus defined relative to the x-axis, and its values are in the interval $[-\pi,\pi]$. However, because of the symmetry of the \wat~and \suldiox~molecules and the equivalence of the two H or O atoms (in the \wat~or\suldiox, respectively), $\phi=-\pi$ is equivalent to $\phi=\pi$, and both situations are equivalent to $\phi=0$. Because $\phi=-\phi$, the value is reported within the range of $[0,\frac \pi 2]$. Figure \ref{fig:spherical-angle} shows the angle definitions using the z-axis as the reference axis.

\begin{figure}[h!]
	\begin{center}
		\includegraphics[scale=1.0]{images/molecularframesmall.png}
		\caption{The body-fixed axes are defined with the x-z plane formed by the three atoms, and the z-axis aligned to the molecular bisector. This sets the y-axis normal to the molecular plane, and one of the bonds in the positive x direction.}
		\label{fig:molecular-frame}
	\end{center}
\end{figure}

\begin{figure}[h!]
	\begin{center}
		\includegraphics[scale=1.0]{images/wateranglessmall.png}
		\caption{The definition of the angles $\theta$ and $\phi$ used to define molecular orientation of \suldiox~and \wat~relative to a reference axis. $\theta$ is the value of the ``tilt'' of the molecular bisector from the reference axis. $\phi$ is the ``twist'' angle that indicates how flat the molecule is relative to the interfacial plane, and is formed between the vector normal to the molecular plane (the body-fixed y-axis) and the reference axis.}
		\label{fig:water-angles}
	\end{center}
\end{figure}

\subsection{Hydrate Complex Analysis}

	During the adsorption process of a gaseous \suldiox~moving in towards a water surface hydrate complexes are formed as the waters bind to the molecule. Computationally, there are various methods to determine hydrate complex formation. The most simple metric is to count the number of hydrogen bonds formed from the \suldiox~to neighboring molecules. This method can be furthered extended to also include the strength of those bonding interactions by additionally reporting the bondlengths to the hydrogen bonding partners. Longer bondlengths are typical of weaker bonds, and this method is used in this work to identify the bonding environment of the \suldiox~during solvation.

	A simple count of hydrogen bonds is telling of the interactions and environments of an adsorbing \suldiox. However, a more sophisticated analysis is needed to fully characterize the geometry of any complexes that form. In this work a graph theoretical method has been used to identify hydrate complexes that form cyclic structures in addition to those that form more simple hydrogen bonding geometries. Previously published works have detailed graph theory for use in molecular systems, and have used graph theoretical methods to describe ice and water clusters, structural energies, as well as in the study of relationships in biological systems.\cite{Huber2007,Shi2005,Radhakrishnan1991,Anick2002} As such, we will not present here a review of the basic graph theoretical details that are well documented in the referenced works. As shown in figure \ref{fig:so2-complex-graph}, the atoms of a \suldiox~and its nearest neighbors are represented as a set of verrtices, with the covalent and hydrogen bonds between them forming the edges of a graph.

	We present here a technique for determining bonding cycles in a hydrate complex, but this technique is general enough to be applied to any graph representation of simple molecular systems. The first step in the determination of cycles is translating the atoms (vertices) and bonds (edges) to a graph. Figure \ref{fig:so2-complex-graph} shows the procedure visually, where here the root vertex of the graph is the \suldiox-sulfur, and each atom is numbered by the distance, in atoms, to the root vertex. We use only the closest neighboring water molecules that may take part in a cyclic complex in building the graph. After forming the graph, a breadth-first search (BFS) is performed starting at the root vertex.\cite{Knuth1997} The BFS algorithm uses a coloring technique where all unexamined vertices are colored white, queued vertices are gray, and examined vertices are black. During the BFS, encountering a gray vertex as a target of an adjacent edge during examination indicates that a cycle has been found in the graph. At each vertex dequeue for examination, the parent of every vertex is recorded such that the lineage from gray target to root vertex can be reconstructed to determine the participating atoms in the cycle.

\begin{figure}[h!]
	\begin{center}
		\includegraphics[scale=1.0]{images/so2complexfigure.png}
		\caption{A \suldiox~and its nearest neighbors are represented as a graph of vertices and edges. The bonds are all given equal weights and the graph is undirected. The numbering shows the relative distance from any atom to the root node (\suldiox-sulfur 1), and also enumerates the iteration through the BFS process at which the atom is discovered. A BFS on this graph would result in discovering the doubly-bound hydrogen (5) as a gray target, and the gray source would be either of the connected oxygens (4). At each iteration the predecessor in the BFS is recorded in order to reconstruct the connectivity of the cycle. This cycle is composed of 8 atoms, and 3 molecules}
		\label{fig:so2-complex-graph}
	\end{center}
\end{figure}
