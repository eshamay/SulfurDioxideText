\section{Steered MD Transit Simulations}

\subsection {\suldiox~Orientation}

	The orientation of a \suldiox~molecule throughout the aqueous adsorption process was monitored during the transit SMD simulations. The angles $\theta$ and $\phi$ (figure \ref{fig:water-angles}) were calculated for each timestep of the SMD simulations as the transiting \suldiox~was pulled into the water slab from the gas phase, both in the neat-water and saturated slab systems. The angle cosine values were collected for the 50 simulations of both systems for each distance from the water surface, resulting in the 2-dimensional histograms shown in figure \ref{fig:so2-transit-angles}. Only the surface into which the transiting \suldiox~was steered into is shown for the saturated system's analysis. In all the data sets, the populations of the angle histograms at each distance from the surface were normalized to aid in comparison of regions with differing \suldiox~residence times.

\begin{figure}[h!]
	\begin{center}
		\includegraphics[scale=1.0]{images/transit-so2-angles/so2-angles-transit.png}
		\caption{\suldiox~molecular orientation distributions during SMD transit simulations into an aqueous slab. Both the neat-water (top row) and saturated (bottom row) data sets were analyzed to determine the angles $\theta$ (left column), and $\phi$ (right column) of the transiting \suldiox. The distributions show angle cosines plotted against the distance of the \suldiox~to the location of the water surface. The data sets were averaged at each distance for the respective 50 simulations, and every distribution at each distance was normalized such that each histogram has a max population of 1 for purposes of comparison.}
		\label{fig:so2-transit-angles}
	\end{center}
\end{figure}

	From its starting position 20\angs above the water surface, until the \suldiox~moves to within 10\angs of the surface of both systems, the orientation is isotropic in $\theta$ and $\phi$. The bisector angle $\theta$, near to the surface becomes more perpendicular ($\cos(\theta)\approx1$) with the \suldiox~sulfur pointing into the water phase. At the point when the \suldiox~reaches the surface location (distance$=0$), the bisector is perpendicular in both the neat-water and saturated systems. There are differences between the neat-water and saturated systems, however, in the onset points of the orientational preferences. In the absence of simulated ionic species that form through \suldiox-\wat~chemistry, it is clear that the adsorbing \suldiox~in the gas phase takes on a preferred orientation to adsorb on a water surface.
 
  % talk about theta differences
  The $\theta$ distribution clearly shows that the \suldiox~is isotropically oriented in the gas-phase and then orients with its bisector perpendicular to the water surface within a certain distance above the water phase. On approach to the neat-water surface, that transition occurs once just above 10\angs from the surface, and again at 5\angs from the surface and below. The onset of orientation at 10\angs appears to be an artifact of the small data set and the normalization scheme, as the isotropic behavior resumes between 5-10\angs. Below 5\angs, however, the orientation persists and the \suldiox~begins to interact with the topmost surface waters, and orients accordingly. In the saturated system, the same trend occurs, however the onset of the perpendicular orientation begins at approx. 8\angs above the water surface. The layer of adsorbed \suldiox~already present in the saturated system most likely interacts with the transiting \suldiox~molecule. Also, topmost water molecules from the surface move up to a few\angs within the \suldiox~layer and interact with the transiting \suldiox~further from the surface than those in the neat-water system.

  % talk about phi differences
  With a mostly perpendicular bisector angle, it is expected that the values of $\phi$ would be isotropic relative to the reference axis. This is the case in both systems with one exception in the neat-water system. At the water surface and just below from 0\angs to 5\angs in the water phase, the $\theta$ profile broadens, and a clear trend appears in the profile of $\phi$. This corresponds to a \suldiox~orienting more flat to the surface (inclined up to 60 degrees from the surface normal). However, below this 5\angs region, the \suldiox~resumes an isotropic orientation as it becomes fully solvated by bulk waters.

  %Upon approaching the saturated surface at approx. 5\angs, the \suldiox~$\phi$ orientation distribution begins to cluster near $\cos(\phi)=1$, with the $\theta$ distribution remaining more isotropic, but with a trend of $\cos(\theta)>0$. At 5\angs the \suldiox~begins interacting with the other \suldiox~molecules that sit atop the water, and the resulting orientation distribution suggests that contact with the \suldiox~layer causes the adsorbing molecule to lie more flat to the surface. Within this same region from 0-5\angs above the surface, the $\phi$ distribution of the neat-\wat~system remains isotropic. The intensity of the distribution (darker blue) also indicates that the \suldiox~spends little time in this region, as it is quickly adsorbed further into the water bulk.

	%Just below the water surface as the distance moves into negative values, both systems show clear orientational preferences. In both systems the bisector orientation clusters such that $\cos(\theta)<0.5$, with most of the distribution intensity around $\cos(\theta)=1$. This suggests that \suldiox~in a water surface orients with the sulfur pointing into the water bulk, and the oxygens pointing towards the surface water molecules.

	%Differences occurr between the system orientation profiles once the \suldiox~has moved further than 10\angs below the surface. While both $\theta$ and $\phi$ hold the same trend from just under the surface through to the bulk of the slab in the saturated system, the orientation profile becomes much more isotropic past 10\angs under the surface of the neat-\wat~system. The orientation effect is shallower under the neat-\wat~surface. However, the orientation distribution peaks most intensely, and more tightly clustered with the \suldiox~bisector aligning with the surface normal, just underneath the neat-\wat~surface than the saturated one. The distributions thus show an orienting force extending deeper into the surface of the saturated system than the neat-\wat~system. In terms of how deeply the \suldiox~molecule is oriented, the interfacial region of the saturated system is larger and extends further into the aqueous bulk.
