\section {Conclusions}

We have presented the results of several classical simulations of \suldiox~interacting with an aqueous surface. We set out to find the orientational behavior of \wat~and \suldiox~through depth-profiling of molecular orientational distributions. In previous experimental studies we posited that water reorients in the presence of \suldiox, and this behavior is one of the steps in formation of gaseous hydrate complexes on water's surface. Spectroscopic techniques can not precisely define a depth profile or orientational distribution of adsorbed species, so we have used molecular dynamics simulations to gain that information.

Our simulations recreated the experimental setup of a recent study by our group,\ref{Ota2011} by introducing a concentrated gas phase of \suldiox~to a water surface. It was found that gaseous \suldiox~quickly adsorbs to the water surface, and continues to bind until a complete surface coverage is reached. Surface waters reorient in the presence of adsorbed \suldiox. The waters at the interface of a neat-water surface tend to lay flatter to the surface than when a saturating layer of \suldiox~is present. The waters interacting with the layer of adsorbed \suldiox~orient more perpendicularly to the interface, and further expose their ``free-OH'' uncoupled bonds for interactions with \suldiox, and hydrate complex formation.

The surface behavior of the \suldiox~molecules was also determined. Experimental surface spectroscopy on the \suldiox~vibrational spectra is not possible because of the nature of the experimental geometry. The spectra of the surface \suldiox~vibrational modes are still unreported. However, through the simulations we are able to characterize the orientational behavior of \suldiox~during and after adsorption. Our equilibrium simulations show that a single \suldiox~molecule, representing a low concentration, has a high surface affinity. This was previously reported by Dang.\ref{Baer2010} At a high \suldiox~concentration, the unreacted \suldiox~moves further out of the water phase as more molecules bind on the surface. The orientation of \suldiox~on the water surface was found to be similar for both low and high concentrations. Those \suldiox~molecules at or below the surface strongly orient with the sulfur atom pointed in towards the water bulk, and the oxygen atoms out towards the gas phase. The \suldiox~slightly above the water surface loses the orientational preference within 10 $\AA$ and those further from the water are more isotropically oriented.

Steered molecular dynamics simulations were used to model the behavior of an adsorbing \suldiox~as it moves from the gas phase above the water down through the surface and into the bulk. The results for the transit through the interface show that in both systems of low and high \suldiox~concentration an adsorbing \suldiox has very similar orientation to those already bound to the water surface. As the \suldiox~approaches the water surface the first contact with water causes it to reorient. Within 5 $\AA$ of the surface the \suldiox~is mostly oriented with its sulfur towards the water phase. The \suldiox~pulled further into the water bulk retains its orientation until it is past the interfacial region and then isotropically orients with the bulk water.

This study is one of several needed to characterize \suldiox~adsorption and behavior on aqueous surfaces. The atmospheric chemistry driving many important reactions occurs on aerosol and small particle surfaces. To further our understanding of gaseous adsorption we plan to report further simulation results of temperature and chemical constituent effects on atmospherically relevant water surfaces.
