\subsection{Equilibrated MD}

Geometric analyses were performed to characterize the net molecular orientation of \wat~and \suldiox~molecules at different depths from the water surface location. At each distance from the surface location, an orientation profile was created for both the \wat~and \suldiox~molecules. The orientation distribution for the angles $\theta$ and $\phi$ at all depths were combined to form 2D intensity plots that show how the molecular orientation distributions change with distance to the surface location. These plots (Figures \ref{fig:water-orientation}, \ref{fig:so2-transit-angles}, and \ref{fig:so2-surface-angles}) allow for a visual interpretation of how the net orientations are affected when moving from the gas phase through the interfacial region and to the surface location, and then further into the aqueous interfacial region and bulk. Both the neat-water, with only a single \suldiox~introduced, and the high-concentration saturated system were analyzed. In the case of the neat-water system, the introduction of a single \suldiox~does not greatly affect molecular orientation of water molecules in the interfacial region. These results of the water orientation are very similar to a neat-water system without any adsorbed solutes (not shown).

The 2D orientational depth profile plots show areas of low intensity in dark blue, and highest intensity in dark red. Regions of the plots where the intensity (coloration) is equally distributed from top to bottom along the entire orientational range are considered isotropic. Likewise, areas of the plot with high intensity over a small orientational range are considered to exhibit an orientational preference at the given depth.  The histograms are arranged with the plots of $\cos(\theta)$ on the left and $\cos(\phi)$ on the right. The surface is located at 0\angs. The angle distributions from both simulated slab surfaces were averaged for all the orientation analyses.

\subsubsection{\wat~Orientation}

The orientation depth-profiles for \wat~are shown in Figure \ref{fig:water-orientation} for both the neat-water (top) and saturated (bottom) systems during the equilibrium MD simulations. The interfacial region for both these calculations and the VSF experiments is defined as the region where molecular orientational anisotropy exists around the surface water location. Referring to Figure \ref{fig:water-orientation}, calculations indicate an interfacial width of approximately 10\angs~for the neat-water system, and approximately 16 \angs~for the saturated system. In both systems the strongest orientational preference is found at the slab surfaces (positions above 0\angs) where the water is furthest towards the gas phase. Previous work on orientational preference of water at air surfaces shows the same trend as our neat-water results.\cite{Walker2006b,Hore2008} The narrow "peninsula" of high intensity that extends in the $\cos(\theta)$ plots to the right of the -5\angs~location shows the overall preference of water to orient at the surface. The $\cos(\phi)$ plots are similar to each other with a narrow region of reorientation, but the effect in the interfacial region is greater in the neat-water system as evidenced by the sharper transition in intensity from blue to red, compared to the saturated system that has a less pronounced intensity change.

The bisector tilt of the water molecules, $\cos(\theta)$, concentrates around $\cos(\theta)=0$ within the first few\angs~above and below the water surface location, becoming progressively isotropic further through the interfacial region and into the water bulk of both systems. As the tilt nears $\cos(\theta)=0$ the \wat~bisector lies within the plane of the surface indicating a water orientation either flat on the surface, or with some amount of ``twist'' sending the OH bonds in towards, or out of the bulk. The value of $\phi$ determines the ``twist'' in this case. Both systems show a defined intensity concentration (dark red) in the distributions around $\cos(\phi)=1$ on the aqueous side of the interfacial region. This results from an orientation of the water's y-axis (normal to the molecular plane) aligned perpendicular to the plane of the water surface. Thus, for both the neat-water and saturated systems, the preferred orientation of water molecules at a distance of 0\angs~ is to lie mostly flat to the plane of the interface with a slight ``twist'' sending one OH bond further into the gas phase, and the other OH towards the water bulk. This result agrees somewhat with a recent air/water study by Fan et al. in which the surface orientations of several water models were analyzed.\cite{Fan2009} They simulated water using non-polarizable models, however, which alters the behavior of water at interfacial regions when compared to the polarizable POL3 model used in our work. The main conclusions are similar in that one OH tends to point further into the air phase than the other, but differs in that this effect is less pronounced with the polarizable model.

Although the plots show overall similarities for both the neat-water and saturated systems, the presence of a layer of adsorbed \suldiox~molecules alters the orientation of those waters furthest into the gas phase. For the saturated solution the resulting orientation of waters above 0\angs, shown in the saturated plot of $\cos(\theta)$ of Figure \ref{fig:water-orientation}, is with a bisector pointing further into the adsorbed \suldiox~gas layer, and both hydrogens pointing outward from the aqueous bulk. The effect is more pronounced further from the water phase, and above 5\angs~the $\cos(\theta)$ distribution is completely centered around $\cos(\theta)=+1$ (see Figure \ref{fig:water-angles}).

%The histograms show overall similarities for both systems in their shapes and intensities from approx. 0\angs~and below. The presence of a saturated layer of adsorbed \suldiox~molecules, however, alters the water orientation, but not necessarily the orientational depth of the interface. In both $\theta$ distributions, the orientations of waters in the bulk region are isotropic until 5\angs~below the surface location. At 5\angs~below the surface the water molecules begin to orient with their bisectors within the plane of the interface (perpendicular to the surface normal, $\cos(\theta)\approx 0$). The corresponding location in the plot of $\phi$ shows that those molecules are also mostly flat to the surface ($\cos(\phi)\approx 1$). Moving further out from the bulk and into the gas phase, the distributions show $\cos(\theta)$ increasing. Waters further out from the bulk have fewer bonding interactions, and orient with their hydrogens more towards the gas phase. The bisector pointing further into the gas phase leads to isotropy in the values of $\phi$.  

The noise in the neat-water plots of Figure \ref{fig:water-orientation} above 5\angs~(manifested as disconnected points of intensity throughout the range of orientations) is a result of fewer waters venturing beyond those extents and thus less data far from the water surface location. This is one indication that waters near a layer of adsorbed \suldiox~venture further above the interface relative to the low \suldiox~concentration, where they can have more \suldiox~interaction with the adsorbed gas molecules. These results show that the interactions with neighboring \suldiox~molecules allow the waters above the surface to orient more perpendicularly to the interface. This is consistent with our recent experimental VSFS studies which showed evidence for the reorienting behavior of water due to the \suldiox~interactions with the topmost surface waters.\cite{Ota2011}

The distribution of $\cos(\phi)$ is more sharply defined (i.e. less isotropic) for the neat-water system than for the saturated one. Waters on the neat surface lie flatter, whereas the presence of the \suldiox~allows a greater range of ``twist'' for those waters in the plane of the interface. The $\phi$ distributions quickly become isotropic above the surface as the bisectors orient more perpendicularly, and below the surface as the bulk water loses any orientational preference.

%Peaks in the distributions of the neat-\wat~system are more clearly pronounced as their intensities are more concentrated and larger than the surrounding area of the profiles. This difference indicates that the transition from the preferred orientation at the water surface has a sharper distinction from the isotropic bulk than in the system with the saturated \suldiox~surface. It appears that the same orientation trend is present in both systems, but the presence of the \suldiox~at the water surface decreases the degree of water orientation at the interface.

\begin{figure}[h!]
	\begin{center}
		\includegraphics[scale=1.0]{images/h2o-angles/h2oangles.png}
		\caption{Molecular orientation histograms of \wat~throughout the surface equilibrated systems. The top surface is located at a distance of 0\angs~with negative distance values located in the bulk of the slab. Shown are the angle distributions for $\theta$ (left column) and $\phi$ (right column) in both the neat-\wat~system (top row) and the saturated system (bottom row). The distributions are normalized to account for the changing number of water molecules at different positions in the system. Regions of high intensity are dark red, and low intensity are dark blue. The scattered points of coloration to the far right of each plot indicates that few waters were located at those positions, and thus few data points were collected.}
		\label{fig:water-orientation}
	\end{center}
\end{figure}


\subsubsection{\suldiox~Orientation}

Orientation distributions of the adsorbed \suldiox~molecules were created during the equilibrium simulations for both the neat-water and saturated systems. Figure \ref{fig:so2-orientation} shows the distributions of $\cos(\theta)$ and $\cos(\phi)$ (arranged similarly to the water orientation distributions plots in Figure \ref{fig:water-orientation}). The \suldiox~orientation data set for the neat-water system is much smaller as only a single \suldiox~molecule was simulated in the bulk. The resulting distribution plots are thus noisier than the corresponding saturated plots with more scattered points of high intensity, but effective comparisons can still be drawn.

In the interfacial region within 0-5\angs~of the water surface location (the approximate distance from the surface that \suldiox~begins interacting with the outer-most waters) the angular distribution of the single \suldiox~(in the neat-water system) is concentrated primarily in $\cos(\theta)>0$. The peak of the distribution occurs at $\cos(\theta)=1$. This indicates that the \suldiox~bisector points out of the water surface, with the sulfur atom pointing towards the aqueous bulk, and the two oxygens pointing into the gas phase. This same distribution occurs in the saturated system for positions below 5\angs. Beyond 5\angs~above the surface both distributions become isotropic. Promixity to the water highly orients the \suldiox~with the sulfur atom pointing in towards the water bulk. Moving further away from the water surface, and interacting less with \wat~molecules allows for greater orientational freedom as exhibited in the isotropy of the distributions above 5\angs. %Because both the neat-water and saturated systems show rather similar distributions, it is possible that the concentration of \suldiox~does not strongly affect the orientation of the \suldiox, unlike the orientations of waters.

The plots of $\phi$ are both isotropic, although the neat-water system plot is quite noisy from the small data set. Because the \suldiox~near the surface is oriented perpendicularly to the interface, the $\cos(\phi)$ orientation is expected to be isotropic. Further from the water surface where the bisector orientation becomes isotropic, the $\cos(\phi)$ distribution also becomes isotropic. For the \suldiox~orientation, the $\phi$ angle does not provide further information regarding the surface behavior.

The plots indicate that \suldiox~orients similarly at both the low and high concentration interfacial environments. The plots of $\phi$ for both concentrations exhibit isotropic distributions. However, the $\theta$ values follow similar trends at both concentrations indicating that the \suldiox~sulfur orients down towards the water phase, with both oxygens pointing away from the surface once the \suldiox~is within approximately 5\angs~of the surface. This behavior continues down to at least 5\angs~below the water surface location, notwithstanding any chemical reactions that may occur that are not simulated using the classical MD techniques utilized in this work.

\begin{figure}[h!]
	\begin{center}
		\includegraphics[scale=1.0]{images/so2-angles/so2-angles.png}
		\caption{Molecular orientation distributions for \suldiox~molecules adsorbed to the water slab surface. Distributions are shown for $\cos(\theta)$ (left column) and $\cos(\phi)$ (right column) for both the neat-water (top) and saturated (bottom) systems. For both systems the $\theta$ distributions show \suldiox~bound to the water surface with the sulfur pointing towards the water slab, and the oxygens pointing to the gas phase. In this configuration the $\phi$ distribution is isotropic because of the water slab's in-plane symmetry.}
		\label{fig:so2-orientation}
	\end{center}
\end{figure}
