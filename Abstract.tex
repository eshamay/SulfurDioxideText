One might expect the high surface tension of water to be a barrier to adsorption of a gas onto the liquid phase, but we know that gaseous adsorption on, and absorption into a water surface is a common phenomenon on this planet. What is not commonly known is how an atmospheric gas such as SO2 and molecules at the water surface can overcome the barrier created by strong water-water surface bonding interactions. What this interplay looks like, the distances from the water surface at which these attractive interactions begin, and how they influence the orientational nature of both SO2 and surface water molecules is the focus of this computational study. The results fill a void in the information about this system existing from previous experimental studies by providing information about the dimensional nature of the gassurface interactions, and the details of how the two species twist and turn orientationally with increased surface interactions. Classical molecular dynamics have been employed in both equilibrium and steered molecular dynamics (SMD) simulations for SO2 at a neat-water surface, and at a surface with high interfacial SO2 concentrations. The results provide new molecular insights for understanding the interaction of this prevalent gas on aerosols and other aqueous surfaces in the environment
