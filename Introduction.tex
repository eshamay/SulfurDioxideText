\section{Introduction}

Interactions between sulfur species and aqueous systems are the current topic of an active area of research. Sulfur dioxide remains an important industrial product, and also enters the environment naturally through terrestrial processes. \suldiox~acts as a major component of atmospheric pollution, and is a precursor to acid rain formation, and cloud nucleation. Its high solubility in water makes \suldiox~ an integral compound in many atmospheric reactions and surface reaction mechanisms. We seek a fuller understanding of the interactions between \suldiox~and water at interfaces because of the central role \suldiox~takes in many atmospheric particle and aerosol reactions.

Our recent experimental works provided insight into the adsorption and reaction of gases in atmospheric aerosols.\cite{Tarbuck2005,Tarbuck2006} We concluded that an \suldiox~surface hydrate complex forms when an aqueous surface is exposed to \suldiox~gas. A computational study by Dang et al.\cite{Baer2010} then made a series of predictions of the specific nature of the hydrated complex through classical and ab initio simulations. That work developed a detailed picture of the nature of the \suldiox~surface complex with water, and related it to the surface water OH vibrational IR spectra.

Our latest experimental work on \suldiox~behavior further expands our understanding of \suldiox~in atmospheric processes.\cite{Ota2011} A temperature study showed that the binding of gaseous \suldiox~to water surfaces at atmospherically relevant cold temperatures is greatly enhanced. The increased binding facilitates the reaction with water to produce dissolved sulfur species. However, the surface binding of unreacted \suldiox~complexes was found to be a completely reversible process. The same study also showed that low pH aqueous solutions, as often found in atmospheric aerosol composition, inhibit the reaction of \suldiox, but do not affect the surface binding of the gas molecules. The resultant experimental spectra suggest that the surface waters reorient in response to the presence of gaseous \suldiox. From the fitting analysis, one possible geometry of water was posited with the interfacial water ``free-OH'' bonds orienting more perpendicularly to the plane of the aqueous surface. 

In this study we look at the orientational depth profiles of water and \suldiox~during and after the adsorption process. Spectroscopic experiments can not probe depth or orientation profiles of surface species in the same detailed manner as computational simulations. Thus we hope to complement the experimental results of previous studies. We determine the effect of introducing \suldiox~to a \wat~surface, and analyze the orientational response of both the \suldiox~and \wat~molecules using equilibrium and steered (SMD) classical molecular dynamics simulations. The results of this study will help to understand the behavior of adsorbing gaseous complexes, and will verify predictions of our recent experiments. Along with other ongoing experiments and computations, we develop a more complete picture of gaseous adsorption and complex formation on an aqueous surface.
