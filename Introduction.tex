\section{Introduction}

The interactions between sulfur species and aqueous systems are currently an active area of chemical research. Sulfur dioxide enters the environment as an important industrial product, and also naturally through terrestrial processes. Atmospheric dust particles and gases have been implicated in the oxidation of \suldiox, and act as reaction surfaces for chemical mechanisms that are still poorly understood.\cite{Baltrusaitis2011,Rubasinghege2010,Li2007} \suldiox~acts as a major component of atmospheric pollution, and is a precursor to acid rain formation, and cloud nucleation. Its high solubility in water makes \suldiox~an integral compound in many aqueous atmospheric reactions, as well. We seek a fuller understanding of the interactions between \suldiox~and water at interfaces because they are central to many atmospheric particle and aerosol reactions.

Our recent experimental works provided insight into the adsorption and reaction of gases in atmospheric aerosols.\cite{Tarbuck2005,Tarbuck2006} We concluded that an \suldiox~surface hydrate complex forms when an aqueous surface is exposed to \suldiox~gas. A computational study by Baer et al.\cite{Baer2010} then made a series of predictions of the specific nature of the hydrated complex through classical and ab initio simulations. That work developed a detailed picture of the nature of the \suldiox~surface complex with water, and related it to the surface water OH vibrational IR spectra.

Our latest experiments on \suldiox~behavior further expand our understanding of \suldiox~in atmospheric processes.\cite{Ota2011} A temperature study showed that the binding of gaseous \suldiox~to water surfaces is greatly enhanced at atmospherically relevant cold temperatures. The increased binding facilitates the reaction with water to produce dissolved sulfur species. However, the surface binding of unreacted \suldiox~complexes was found to be a completely reversible process. The same study also showed that low pH aqueous solutions, as often found in atmospheric aerosol composition, inhibit the bulk reaction of \suldiox, but do not affect the surface binding of the gas molecules. The resultant experimental spectra suggest that the surface waters reorient in response to the presence of gaseous \suldiox. From the fitting analysis, a geometry of surface waters was posited with the interfacial water ``free-OH'' bonds orienting more perpendicularly to the plane of the aqueous surface. 

In this study we look at computed orientational depth profiles of water and \suldiox~before and after the adsorption process. Spectroscopic experiments can not probe depth or orientation profiles of surface species in the same detailed manner as computational simulations. Thus we hope to augment the experimental results of previous studies. We determine the effect of introducing \suldiox~to a \wat~surface, and analyze the orientational response of both the \suldiox~and \wat~molecules using equilibrium and steered (SMD) classical molecular dynamics simulations. The results of this study will help to understand the behavior of adsorbing gaseous complexes, and will verify predictions of our recent experiments. Computational simulations like this are necessary to further understand the microscopic nature of the gaseous adsorption process, as many of the molecular-level subtleties are lost in experiment. Along with other ongoing experiments and computations, we develop a more complete picture of gaseous adsorption and complex formation on an aqueous surface.
