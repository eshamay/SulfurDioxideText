\section{Introduction}

Interactions between sulfur species and aqueous systems are the current topic of an active area of research. Sulfur dioxide remains an important industrial product, and also enters the environment naturally through terrestrial processes. \suldiox~acts as a major component of atmospheric pollution and is a precursor to acid rain formation, and cloud nucleation. Its high solubility in water makes \suldiox~ an integral compound in many atmospheric reactions and surface reaction mechanisms. We seek a fuller understanding of the interactions between \suldiox~and water interfaces because of the role \suldiox~takes in atmospheric particle and aerosol chemistry.

Our recent experimental works provided insight into the adsorption and reaction of gases in atmospheric aerosols. We concluded that an \suldiox~surface complex forms with water when an aqueous surface is exposed to \suldiox~gas. A computational study by Dang et al.\cite{Baer2010} then made a series of predictions of the specific nature of the surface complex through classical and ab initio simulations. That work probed the cause of changes in the surface water OH vibrational spectra, and developed a detailed picture of the nature of the \suldiox~surface complex with water.

Our latest experimental work on \suldiox~behavior further expands our understanding of \suldiox~in atmospheric processes.\cite{Ota2011} A temperature study showed that atmospherically relevant cold temperatures greatly enhance the binding of gaseous \suldiox~to water interfaces, and facilitates the reaction with water to produce dissolved sulfur species. However, the surface binding of unreacted \suldiox~complexes was found to be a completely reversible process. The same study also conluded that low pH aqueous solutions, as often found in atmospheric aerosol composition, inhibit the reaction of \suldiox, but does not affect the binding of the gaseous molecules. The analysis of the resultant experimental spectra suggested that the surface waters reorient in response to the presence of gaseous \suldiox~complexes, with the interfacial water ``free-OH'' bonds orienting more perpendicularly to the plane of the aqueous surface. Because of the geometry of the experiment the \suldiox~modes were not possible to analyze.

In this study we look at the orientational depth profiles of water and \suldiox~during, and after, the adsorption process. Spectroscopic experiments can not probe depth or orientation profiles of surface species in the same detailed manner as computational simulations. Thus we hope to complement the experimental results of previous studies. We determine the effect of introducing \suldiox~to a \wat~surface on both the \suldiox~and \wat~molecules using classical molecular dynamics simulations. The results of this study are necessary to understand the behavior of adsorbing gaseous complexes, and we help verify the ascertions of our recent experiments. Along with other ongoing studies, we develop a more complete and detailed picture of gaseous adsorption and complex formation on aqueous surface.
