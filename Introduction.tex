\section{Introduction}

The doorway to the uptake of a gas by an aqueous solution is the water surface. Although we know much about the behavior of a gas on either side of that entrance, far less is known about how that surface acts to attract, facilitate, or thwart the transit of a molecule between the two bulk phases. What is the interplay between the gas and surface water molecules, and when does one begin to influence the behavior of the other? What species form during gas adsorption onto liquid surfaces, and what are the intermediary steps?  Is molecular orientation of either the gas or surface molecules a factor in the adsorption process? Are specific gas or liquid molecular orientations necessary for gaseous adsorption?  Experimental studies to address such questions are valuable but do not provide the full resolution necessary to determine the geometries of adsorbing gases, or to determine the orientations of the molecules at the liquid surface near the adsorption site. This type of information can be determined computationally, and when coupled to the experimental studies can provide a more comprehensive picture of the gas-liquid surface adsorption process.

An important gas for developing a picture of gaseous adsorption and entry into a water surface is sulfur dioxide.\cite{Donaldson1995,Lattanzi2010,Shah2011,Tzivian2011,Johns2011,Faloona2009,Jurkat2010,Wu2011,Jayne1990,Jayne1990a,Yang2002} \suldiox~enters the environment as an important industrial product, and also naturally through terrestrial processes. Atmospheric dust particles and gases have been implicated in the oxidation of \suldiox, and act as reaction surfaces for chemical mechanisms that are still poorly understood.\cite{Baltrusaitis2011,Rubasinghege2010,Li2007,Madsen2008,Boniface2000} \suldiox~acts as a major component of atmospheric pollution, and is a precursor to acid rain formation, and cloud nucleation. Its high solubility in water makes \suldiox~an integral compound in many aqueous atmospheric reactions, as well.  Obtaining a more complete picture of the \suldiox~adsorption process is important for understanding gaseous adsorption of this environmentally important gas on water and aerosol surfaces as well as being a model system for understanding the more general nature of gases at aqueous interfaces.

In this work we provide a molecular picture of \suldiox~adsorption on a water surface.  This study demonstrates the strong orientational effect of surface water molecules on the adsorbing gas during the approach and entry into the surface region at both high and low \suldiox~surface concentrations.  These computational studies complement and significantly expand the picture developed in our recent experimental vibrational sum frequency spectroscopy (VSFS) studies of \suldiox~adsorption of aqueous solutions of various compositions and temperatures,\cite{Tarbuck2005,Tarbuck2006} and the subsequent studies using both classical and ab initio simulations.  These experimental studies showed that an \suldiox~surface hydrate complex forms when an aqueous surface is exposed to \suldiox~gas. The computational study by Baer et al.\cite{Baer2010} then made a series of predictions of the specific nature of the hydrated complex through classical and ab initio simulations. That work developed a detailed picture of the nature of the \suldiox~surface complex with water, and related it to the surface water OH vibrational IR spectra.  The most recent experimental studies have shown that whereas the binding of gaseous \suldiox~to a water surface is greatly enhanced at cold temperatures, the reversibility of the adsorption process remains.\cite{Ota2011}  Complementary experiments showed that low pH aqueous environments inhibit the bulk reactions of \suldiox, but do not affect the surface binding or its reversibility.  What is apparent in the VSF spectra obtained in all of these experiments is the tendency of water to reorient upon surface bonding, with the effect becoming more pronounced at high \suldiox~surface concentrations. Since the \suldiox~molecule was not specifically probed, conclusions on how \suldiox~bonding contributes to reorienting surface water molecules and the orientation of \suldiox~itself upon approach and surface bonding could only be inferred. 

To fill this void, the computational studies described herein provide a detailed picture of the orientation of both \suldiox~and surface water molecules during the adsorption process.  The depth profiling studies which examine the orientation of both species during the approach and entry of the gas into the interfacial region are obtained using equilibrium and steered (SMD) classical molecular dynamics simulations. The latter approach involves steering a gas molecule into the aqueous phase, and characterizing its molecular orientation as it transits through the interfacial region.  This unique approach enables new insights into the behaviors of gas molecules as they move near to liquid water. We also simulate how \suldiox~adsorption occurs on a water surface saturated with adsorbed \suldiox, analogous to the conditions of the \suldiox~experimental studies recently performed.\cite{Ota2011} The results of this study provide an intimate perspective on the adsorption of \suldiox~at an aqueous surface, and a more complete picture of gaseous adsorption to liquid interfaces.
