\section{Introduction}

The doorway to the uptake of a gas by an aqueous solution is the water surface. Although we know much about the behavior of a gas on either side of that entrance, far less is known about how that surface acts to attract, facilitate, or thwart the transit of the gas between the two bulk phases. What species form during gas adsorption onto liquid surfaces, and what are the intermediary steps?  To what extent does an adsorbing gas affect liquid water surface molecules, or the structure of the interface throughout an adsorption process?  Are specific gas or liquid molecular orientations necessary for gaseous adsorption?  Experimental studies to address such questions are valuable but do not provide the resolution necessary to determine the geometries of adsorbing gases, or to determine the orientations of the molecules at the liquid surface near the adsorption site.
This type of information can be determined computationally, and when coupled to the experimental studies can provide important new insight into the gas-liquid suface adsorption process.

In this work we look at the orientational properties of sulfur dioxide gas adsorbing to a water surface as a model for the transit of gases across an aqueous surface.  The interactions between sulfur species and aqueous systems are currently an active area of chemical research. Sulfur dioxide enters the environment as an important industrial product, and also naturally through terrestrial processes. Atmospheric dust particles and gases have been implicated in the oxidation of \suldiox, and act as reaction surfaces for chemical mechanisms that are still poorly understood.\cite{Baltrusaitis2011,Rubasinghege2010,Li2007} \suldiox~acts as a major component of atmospheric pollution, and is a precursor to acid rain formation, and cloud nucleation. Its high solubility in water makes \suldiox~an integral compound in many aqueous atmospheric reactions, as well. We seek a fuller understanding of the interactions between \suldiox~and water at interfaces because they are central to many atmospheric particle and aerosol reactions.

Our recent vibational sum frequency spectroscopy (VSFS) experiments have provided insight into the adsorption and reaction of gases in atmospheric aerosols.\cite{Tarbuck2005,Tarbuck2006} These studies showed that an \suldiox~surface hydrate complex forms when an aqueous surface is exposed to \suldiox~gas. A computational study by Baer et al.\cite{Baer2010} then made a series of predictions of the specific nature of the hydrated complex through classical and ab initio simulations. That work developed a detailed picture of the nature of the \suldiox~surface complex with water, and related it to the surface water OH vibrational IR spectra.

Our latest VSFS experiments on \suldiox~behavior further expand our understanding of \suldiox~in atmospheric processes.\cite{Ota2011} A temperature study has shown that the binding of gaseous \suldiox~to water surfaces is greatly enhanced at atmospherically relevant cold temperatures. The surface binding of unreacted \suldiox~complexes is found to be a completely reversible process. The same study also showed that low pH aqueous environments inhibit the bulk reaction of \suldiox, but do not affect the surface binding of the gas molecules. Furthermore, the resultant experimental spectra suggest that the surface waters reorient in response to the presence of adsorbed \suldiox, but the details about this reorientation can not be garnered from the experiments.  Such details include knowing how surface water orientation changes upon approach of the gaseous \suldiox, how this varies upon \suldiox~surface bonding, and the orientation of \suldiox~throughout the surface bonding process. 
%Knowing that reorientation may occur due to \suldiox~adsorption, how can we confirm it? what is the nature of the reorienting waters? How do they reorient, and to what extent? Conclusive experimental evidence of the adsorbing gas or surface water molecular orientation is still missing, so we are moved to use computational methods for further understanding this phenomena.

We provide such detailed informatino to complement and augment these experimental studies. Our studies involve computation of the orientational depth profiles of water and \suldiox~before and after the adsorption process. They augment the spectroscopic experiments that can not probe depth or orientation profiles of surface species in the same detailed manner as computational simulations. The molecular behaviors that occur upon introduction of \suldiox~to a \wat~surface are determined, and we analyze the orientational response of both the \suldiox~and \wat~molecules using equilibrium and steered (SMD) classical molecular dynamics simulations. The latter approach involves steering a gas molecule into the aqueous phase, and characterizing its molecular orientation as it transits through the interfacial region.  This unique approach enables new insights into the behaviors of gas molecules as they move near to liquid water to be obtained. We also simulate how \suldiox~adsorption occurs on a water surface saturated with adsorbed \suldiox, analogous to the cold \suldiox~experimental studies recently performed.\cite{Ota2011} The results of this study provide an intimate perspective on the adsorption of \suldiox~at an aqueous surface, and a more complete picture of gaseous adsorption to liquid interfaces.
