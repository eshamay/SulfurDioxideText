\section{Introduction}

The physics of gaseous adsorption on water surfaces is a mostly unexplored field. Our current molecular-level understanding of adsorption phenomena relies on experiments and theory of model systems, and on an increasing number of computational simulations of molecular processes. What species form during gas adsorption onto liquid surfaces, and what are the intermediary steps? To what extent does an adsorbing gas affect liquid water surface molecules, or the structure of the interface throughout an adsorption process? Are specific gas or liquid molecular orientations necessary for gaseous adsorption? Very little work has been done to characterize the molecular processes that take place during the transit of a molecule from the gas phase onto a liquid interfacial region. Furthermore, experiment does not yet provide us with the resolution necessary to determine the geometries of adsorbing gases, or to determine the orientations of the molecules at the liquid surface near the adsorption site. In this work we look at the orientational properties of sulfur dioxide gas adsorbing to a water surface as a model for the transit of gases to liquid interfaces.

The interactions between sulfur species and aqueous systems are currently an active area of chemical research. Sulfur dioxide enters the environment as an important industrial product, and also naturally through terrestrial processes. Atmospheric dust particles and gases have been implicated in the oxidation of \suldiox, and act as reaction surfaces for chemical mechanisms that are still poorly understood.\cite{Baltrusaitis2011,Rubasinghege2010,Li2007} \suldiox~acts as a major component of atmospheric pollution, and is a precursor to acid rain formation, and cloud nucleation. Its high solubility in water makes \suldiox~an integral compound in many aqueous atmospheric reactions, as well. We seek a fuller understanding of the interactions between \suldiox~and water at interfaces because they are central to many atmospheric particle and aerosol reactions.

Our recent vibational sum frequency spectroscopy (vsfs) experiments provided insight into the adsorption and reaction of gases in atmospheric aerosols.\cite{Tarbuck2005,Tarbuck2006} We concluded that an \suldiox~surface hydrate complex forms when an aqueous surface is exposed to \suldiox~gas. A computational study by Baer et al.\cite{Baer2010} then made a series of predictions of the specific nature of the hydrated complex through classical and ab initio simulations. That work developed a detailed picture of the nature of the \suldiox~surface complex with water, and related it to the surface water OH vibrational IR spectra.

Our latest vsfs experiments on \suldiox~behavior further expand our understanding of \suldiox~in atmospheric processes.\cite{Ota2011} A temperature study showed that the binding of gaseous \suldiox~to water surfaces is greatly enhanced at atmospherically relevant cold temperatures. The surface binding of unreacted \suldiox~complexes was found to be a completely reversible process. The same study also showed that low pH aqueous environments inhibit the bulk reaction of \suldiox, but do not affect the surface binding of the gas molecules. The resultant experimental spectra suggest that the surface waters reorient in response to the presence of adsorbed \suldiox. Knowing that reorientation may occur due to \suldiox~adsorption, how can we confirm it? what is the nature of the reorienting waters? How do they reorient, and to what extent? Conclusive experimental evidence of the adsorbing gas or surface water molecular orientation is still missing, so we are moved to use computational methods for further understanding this phenomena.

In this study we look at computed orientational depth profiles of water and \suldiox~before and after the adsorption process. Spectroscopic experiments can not probe depth or orientation profiles of surface species in the same detailed manner as computational simulations. We determine the effect of introducing \suldiox~to a \wat~surface, and analyze the orientational response of both the \suldiox~and \wat~molecules using equilibrium and steered (SMD) classical molecular dynamics simulations. We use a technique of steering a gas molecule into the aqueous phase, and characterize the transit through the interfacial region. This unique method gives us much greater insight into the behaviors of gas molecules as they move near to liquid water. We also simulate the behavior of adsorbed \suldiox~at equilibrium to show how an already-adsorbed \suldiox~behaves on a water surface. The results of this study will help us to understand the behavior of adsorbing gases. Computational simulations like this are necessary to further understand the microscopic nature of the gaseous adsorption process, as many of the molecular-level subtleties are lost in experiment. Along with other ongoing experiments and computations, we develop a more complete picture of gaseous adsorption on aqueous surfaces.
